\documentclass{report}
\usepackage{caption}
\usepackage{indentfirst}
\usepackage{float}
\usepackage{listings}
\usepackage{footnote}
\usepackage[hidelinks]{hyperref}
\usepackage{multirow}
\usepackage{graphicx} % Required for the inclusion of images
\usepackage{cite}
\usepackage{url}
\usepackage[utf8]{inputenc}
\usepackage[T1]{fontenc}
\usepackage{listings}
\renewcommand\thechapter{\arabic{chapter}}
\renewcommand\thesection{\arabic{section}}
\renewcommand\contentsname{Tabela de Conteúdos}
\PassOptionsToPackage{hyphens}{url}\usepackage{hyperref}

\title{\textit{Information-Centric Networking}:\\A Implementação através de \textit{CCNx Project}} % Title

\author{André Diegues - 201206858 \\ Fábio Teixeira - 201305725\\ % Author name
		T\'{o}picos Avan\c{c}ados em Redes - CC4037\\
		Departamento de Ciencia de Computadores\\
		Faculdade de Ciencias da Universidade do Porto
}
\date{22 de Dezembro de 2016} % Date for the report

\begin{document}

\maketitle % Insert the title, author and date
\tableofcontents

\section{\textit{CCNx Project}}

O \textit{CCNx Project} foi desenvolvido pelo PARC\footnote{\url{https://www.parc.com}} e a sua implementação corresponde à definição de \textit{Content-Centric Networking}\cite{content}. A concretização do \textit{CCNx Project} promove uma maneira nova de comunicar na \textit{Internet}, nomeadamente através de conteúdo. \\

A arquitetura do \textit{CCNx Project} tem definidos vários módulos, sendo os principais o \textit{naming} do conteúdo, mensagens de rede (\textit{Interest, Content Object e InterestReturn}), \textit{Matching} desses conteúdos, verificação e encaminhamento\cite{ccnx}. \\

O \textit{CCNx Distillery} é a segunda versão do \textit{CCNx Project} que liga estes módulos e monta-os de forma a criar o \textit{software} funcional. A maior alteração desta versão para a sua precedente é que agora os módulos são independentes uns dos outros, desta forma tornou-se mais eficiente. \\

O \textit{CCNx Project} tem implementado os protocolos necessários para os interessados nesta vertente de comunicação \textit{Internet} poderem fazer experiências, pesquisas e desenvolver sobre a tecnologia.\\

O protocolo base utiliza dois tipos de mensagem:

\begin{itemize}
\item \textit{Interest}\\

Serve para pedir um \textit{Content Object} à rede e é encaminhado através das rotas que conhece.\\

\item \textit{Content Object}\\

Serve para enviar o conteúdo requerido por mensagens \textit{Interest} provenientes da rede. Este conteúdo tem de ser autenticado devidamente pelo utilizador que dispõe o conteúdo.\\

\end{itemize}

Uma rede CCNx contém\cite{introccnx} \textit{Content Producers}, \textit{Content
Publishers}, \textit{Content Consumers}, \textit{Caches} e \textit{Forwarders}.\\

Como os próprios nomes indicam, um \textit{Content Producer} produz conteúdo, um \textit{Content Publisher} pega nesse conteúdo produzido e transforma-o em \textit{Content Objects} com a autenticidade do \textit{Content Producer} e um \textit{Content Consumer} pede o conteúdo através do protocolo CCNx com uma mensagem \textit{Interest}. O encaminhamento de mensagens é feito através do \textit{Forwarder} enquanto que os \textit{in-network caches} armazenam os \textit{Content Objects} temporariamente (um tempo que é especificado pelo \textit{Content Publisher} no momento que cria o \textit{Content Object}).\\

\section{Objetivos}

%If you have more than one objective, uncomment the below:
\begin{description}
\item[Entender o estado da arte] \hfill \\

O tráfego na \textit{Internet} hoje em dia corresponde maioritariamente a transmissão de conteúdo entre \textit{hosts}. Apesar da atual arquitetura já ter soluções para este tipo de comunicação não é uma arquitetura que está preparada para que o tráfego de conteúdo cresça como tem vindo a crescer ao longo dos anos e, neste aspeto, uma arquitetura \textit{data-centric} baseada no ICN, como o CCNx, seria muito vantajosa\cite{icntar}.\\

O CCNx procura substituir a arquitetura da \textit{Internet} - um modelo de comunicação \textit{host-to-host} - por uma arquitetura baseada num modelo \textit{data-centric}, tratando o conteúdo como entidade principal na arquitetura das redes. Uma rede com este tipo de arquitetura ganha inúmeras vantagens em relação ao modelo \textit{host-to-host}, nomeadamente, na distribuição de conteúdo, segurança e desenvolvimento de aplicações\cite{icntar}.\\

Gostaríamos, com este trabalho laboratorial, de perceber se a mudança de arquitetura seria algo fácil e praticável ou se, por outro lado, esta tecnologia ainda necessita de mais tempo para se tornar numa opção credível à arquitetura atual.\\

\item[Executar um exemplo simples] \hfill \\

O objetivo fulcral deste trabalho laboratorial é ver como funciona uma rede baseada em ICN, como tal, gostariamos de pôr em prática primeiro um exemplo simples de forma a estudar como se comporta a rede e as suas interações. A realizar este objetivo também concretizámos o objetivo de criar uma rede ICN.\\


\item[Aplicação a um caso real] \hfill \\

Objective 3 text\\
\end{description}

\section{Configuração do Sistema}

A configuração do sistema dividiu-se em duas importantes partes. A primeira foi a instalação do CCNx, já a segunda prendeu-se com o estabelecimento da interação com o módulo que faz o encaminhamento na rede, o Metis. Para isso, guiámo-nos por instruções básicas para instalar o \textit{CCNx Binary Release}\footnote{\url{http://blogs.parc.com/ccnx/howto-install-ccnx-binary-release/}}.\\

\subsection{\textit{CCNx Project}}

Primeiramente, foi necessário fazer um \textit{update} do Ubuntu e instalar algumas bibliotecas:\\

\noindent{\texttt{sudo apt-get update -y \\
apt-get install uncrustify doxygen -y\\
apt-get install libssl-dev openssl libevent-dev -y\\}}

Com isto, o \textit{hardware} e o Sistema Operativo ficaram preparados para instalar o \textit{software} do CCNx, distribuído pelo seu domínio oficial (\url{CCNx.org}), na forma de um ficheiro zipado, com extensão .tgz. Para ter acesso ao ficheiro zipado, usou-se o \texttt{curl}:\\

\noindent{\texttt{curl -O http://www.ccnx.org/releases/distillery-ccnx-absinthe-\\1.0.20150516-Linux-x86\_64.tgz\\}}

E para o extrair, o \texttt{tar}:\\

\noindent{\texttt{sudo tar -xzf distillery-ccnx-absinthe-1.0.20150516-Linux-x86\_64.\\tgz -C /\\}}

Para finalizar a instalação do CCNx, bastou correr o distillery-post-install, presente no diretório bin, do ccnx:\\

\noindent{\texttt{/usr/local/ccnx/bin/distillery-post-install\\}}

E, assim, o CCNx ficou pronto para ser utilizado. Para facilitar o tratamento de caminhos pelos ficheiros, entregou-se a algumas variáveis pequenos \textit{paths}, utilizando o \texttt{export}:\\

\noindent{\texttt{export PARC\_HOME=/usr/local/parc\\
export CCNX\_HOME=/usr/local/ccnx\\
export PATH=\$PATH:\$PARC\_HOME/bin:\$CCNX\_HOME/bin\\
export LD\_LIBRARY\_PATH=\$LD\_LIBRARY\_PATH:\$PARC\_HOME/lib:\$CCNX\_HOME\\/lib\\}}

No entanto, mais à frente, optou-se por uma versão otimizada do CCNx, com melhorias de \textit{performance} mas com maiores dificuldades em fazer \textit{debugging}. Para isso, fez-se \textit{download} do seguinte \textit{link}:\\

\noindent{\texttt{wget http://ccnx.org/releases/distillery-ccnx-absinthe-\\1.0.20150516-Linux-x86\_64.optimized.tgz\\}}

Sendo que posteriormente, foi preciso extraí-lo:\\

\noindent{\texttt{sudo tar -xzf distillery-ccnx-absinthe-1.0.20150516-Linux-x86\_64.\\optimized.tgz -C /\\}}

E novamente, para instalar, recorrer ao mesmo ficheiro:\\

\noindent{\texttt{/usr/local/ccnx/bin/distillery-post-install\\}}

Uma vez feito isto, foi tempo de correr o \textit{daemon} responsável por fazer o \textit{forwarding} do CCNx, com \textit{root} iniciado, que iria depois ficar em \textit{background}:\\

\subsection{Metis}

 \noindent{\texttt{wget http://glaros.dtc.umn.edu/gkhome/fetch/sw/metis/metis-5.1.0.tar.gz\\
     gunzip metis-5.1.0.tar.gz \\
     tar -xvf metis-5.1.0.tar \\
     cd metis-5.1.0\\
     apt-get install cmake\\
     apt-get update \&\& apt-get install build-essential\\
     make config\\
     make\\
     make install\\}}


\section{Dados Experimentais}

\subsection{Exemplo simples}

\noindent{\texttt{metis\_daemon --capacity 0 \&\\}}

O passo seguinte foi criar uma chave, com um limite de 2048 bits e uma validade de 1000 dias, para depois usar nos pedidos:\\

\noindent{\texttt{parc\_publickey create .ccnx\_keystore.p12 <password> <username> \\2048 1000\\}}

Neste caso foi necessário substituir os campos \texttt{<password>} e \texttt{<username>} por algo específico, característico do utilizador. Para que um \textit{Publisher} colocasse algo na rede, fez-se:\\

\noindent{\texttt{ccnx\_server --identity .ccnx\_keystore.p12 --password <password> \\lci:/date /bin/date \&\\}}

Sendo que o processo também ficou em \textit{background}. Representativamente, o que está anexado ao "\texttt{lci:}", simboliza o nome do ficheiro que fica na rede, e o que vem depois, a localização do ficheiro no diretório corrente. Para um \textit{Consumer} pedir conteúdo na rede (neste caso o \texttt{/date}), fez-se:\\

\noindent{\texttt{ccnx\_client --identity .ccnx\_keystore.p12 --password <password> \\lci:/date\\}}

Uma vez que os processos do ccnx\_server e do metis ficaram em \textit{background}, é gerada imediatamente uma mensagem sempre que um \textit{Consumer} pede por conteúdo na rede. Para ver (e listar) a tabela de \textit{routing} usada pelo \textit{daemon}, fez-se:\\

\noindent{\texttt{ccnx\_client --identity .ccnx\_keystore.p12 --password <password> \\lci:/date\\}

\subsection{Aplicação a um caso real}

\section{Resultados e Conclusões}

\bibliographystyle{plain}

\bibliography{sample}

%----------------------------------------------------------------------------------------


\end{document}