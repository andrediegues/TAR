\documentclass[conference]{IEEEtran}

\usepackage{cite}

% correct bad hyphenation here
\hyphenation{op-tical net-works semi-conduc-tor}


\begin{document}
\title{Information Centric Networking}

\author{
	\IEEEauthorblockN{
		Andr\'{e} Diegues - 201206858\\
		F\'{a}bio Teixeira - 201305725
	}

	\IEEEauthorblockA{
		T\'{o}picos Avan\c{c}ados em Redes - CC4037\\
		Departamento de Ciencia de Computadores\\
		Faculdade de Ciencias da Universidade do Porto
	}
}

\maketitle

\begin{abstract}
Hoje em dia, o enorme aumento de tr\'{a}fego de conte\'{u}do e dados entre utilizadores na Internet motivou o desenvolvimento de v\'{a}rias arquiteturas de \textit{Internet} que resolvam eficazmente este problema. Uma delas \'{e} a que vamos abordar neste artigo, a \textit{Information Centric Networking} (ICN), que atrav\'{e}s de uma abordagem de pesquisa de informa\c{c}\~{a}o nas redes permite fornecer \`{a} rede um servi\c{c}o mais resiliente a falhas que cumpre as exig\^{e}ncias de distribui\c{c}\~{a}o de conte\'{u}do\cite{ahlgren}. Vamos abordar o seu funcionamento, o custo da sua implementa\c{c}\~{a}o e estudar se esta mudan\c{c}a de arquitetura \'{e} ou n\~{a}o vi\'{a}vel. 
\end{abstract}

\section{Introdu\c{c}\~{a}o}
A arquitetura ICN foi baseada numa primeira abordagem de arquitetura denominada de TRIAD\cite{ahlgren}, cujo principal objectivo seria facilitar e aliviar cerca de 80\% do tr\'{a}fego de \textit{Internet} que servia apenas para entrega de conte\'{u}do. A TRIAD define uma nova camada de conte\'{u}do que est\'{a} implementada por \textit{content routers} que encaminham os pedidos aos \textit{content servers} que, de seguida, fornecem o conte\'{u}do\cite{triad}.\\


A ICN procura substituir a arquitetura atual, que \'{e} um modelo de comunica\c{c}\~{a}o \textit{host-to-host}, por uma arquitetura baseada num modelo \textit{data-centric}, tratando o conte\'{u}do como entidade principal na arquitetura das redes. Uma rede com este tipo de arquitetura ganha in\'{u}meras vantagens em rela\c{c}\~{a}o ao modelo \textit{host-to-host}, nomeadamente, na distribui\c{c}\~{a}o de conte\'{u}do, seguran\c{c}a e desenvolvimento de aplica\c{c}\~{o}es\cite{icn}.



\section{Como funciona a ICN?}
Subsubsection text here.


% Note that \label must occur AFTER (or within) \caption.
% For figures, \caption should occur after the \includegraphics.
% Note that IEEEtran v1.7 and later has special internal code that
% is designed to preserve the operation of \label within \caption
% even when the captionsoff option is in effect. However, because
% of issues like this, it may be the safest practice to put all your
% \label just after \caption rather than within \caption{}.

%\begin{figure}[!t]
%\centering
%\includegraphics[width=2.5in]{myfigure}
% where an .eps filename suffix will be assumed under latex, 
% and a .pdf suffix will be assumed for pdflatex; or what has been declared
% via \DeclareGraphicsExtensions.
%\caption{Simulation results for the network.}
%\label{fig_sim}
%\end{figure}

%\begin{figure*}[!t]
%\centering
%\subfloat[Case I]{\includegraphics[width=2.5in]{box}%
%\label{fig_first_case}}
%\hfil
%\subfloat[Case II]{\includegraphics[width=2.5in]{box}%
%\label{fig_second_case}}
%\caption{Simulation results for the network.}
%\label{fig_sim}
%\end{figure*}
%
%\begin{table}[!t]
%% increase table row spacing, adjust to taste
%\renewcommand{\arraystretch}{1.3}
% if using array.sty, it might be a good idea to tweak the value of
% \extrarowheight as needed to properly center the text within the cells
%\caption{An Example of a Table}
%\label{table_example}
%\centering
%% Some packages, such as MDW tools, offer better commands for making tables
%% than the plain LaTeX2e tabular which is used here.
%\begin{tabular}{|c||c|}
%\hline
%One & Two\\
%\hline
%Three & Four\\
%\hline
%\end{tabular}
%\end{table}



\section{O que j\'{a} foi testado?}

\section{Implementar a ICN}

\section{A ICN pode fazer parte do futuro da \textit{Internet}?}


\section{Conclus\~{a}o}
The conclusion goes here.







% trigger a \newpage just before the given reference
% number - used to balance the columns on the last page
% adjust value as needed - may need to be readjusted if
% the document is modified later
\IEEEtriggeratref{8}
% The "triggered" command can be changed if desired:
\IEEEtriggercmd{\enlargethispage{-5in}}

% references section

% can use a bibliography generated by BibTeX as a .bbl file
% BibTeX documentation can be easily obtained at:
% http://mirror.ctan.org/biblio/bibtex/contrib/doc/
% The IEEEtran BibTeX style support page is at:
% http://www.michaelshell.org/tex/ieeetran/bibtex/
\bibliography{icn}
\bibliographystyle{IEEEtran}
% argument is your BibTeX string definitions and bibliography database(s)
%\bibliography{IEEEabrv,../bib/paper}
%
% <OR> manually copy in the resultant .bbl file
% set second argument of \begin to the number of references
% (used to reserve space for the reference number labels box)


% that's all folks
\end{document}


